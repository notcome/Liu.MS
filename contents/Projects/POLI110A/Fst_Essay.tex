\docclass{article}
\import{LiuMS.Texy.General}

\title{First Essay}

This essay is about the Peloponnesian War. The only text I was allowed to draw
reference from is Thucydides’ \workname{History of the Peloponnesian War}. I
could find the original prompts given by the professor, yet I wrote a summary
when writing this essay:

\bgn{\quote}
This essay first discusses how in Greek cities, particularly Athens, the norm of
pursing honor made honor different from a moral standard and more similar to
private interests like wealth in many perspectives. It then presents how this
type of honor both escalated the level of the war and created lot of civil
conflicts in cities. It also compares how different kind of regimes, namely
oligarchy and democracy, reacts to such norm.
\end{\quote}

\hrule

\bgn{\section{The norm of pursing “honor”}}
Honor was intended by leaders of each city as a way to ensure citizens’ loyalty.
In his funeral oration\sup{II.35 – 41}, Pericles first stated\sup{II.37 – 41}
that Athens was a great city, which was the reason why the Athenians should
protect it. He then praised\sup{II.42} the dead, encouraged\sup{II.43} his
fellow citizens to emulate those honorable men as well as giving the reason why
cowardice should be preferred to death for “a man with pride”. He urged
\sup{II.44} that the Athenians should meet the honor and enjoy a life “both the
living and leaving” of which are filled with happiness.

It’s clear that Pericles thinks one should pursue honor and that honor is to
fight for the city—not merely protecting it but also expanding it, as mentioned
in his praise\sup{II.36} for the Athenian ancestors. This should be a widely
accepted idea in Athens, because later in Pericles’ final speech
\sup{II.60 – 64}, he criticized his fellow citizens turning their anger to him
“in the shock of misfortunes in your own homes” rather than placing public
interests over private ones, and people did agree with him.

Ideally, honor should serve as a mental standard and guided the Athenian
citizens to do good for their city. It should be something inner, something that
people obey unconditionally even though they gain nothing from it directly. This
is clearly not true, otherwise people would not blame Pericles after the plague.
They had consented to the war not for the city’s good but for Pericles had
promised that this would be a easy war\sup{I.141 – 143}; when the war turned out
to be harsher than expected, they regretted their original decision.

In contrast, the Athenians started to pursue honor for their personal interests,
like doing just things to gain prestige in contemporary world. \emph{What
matters is not doing honorable things but the honor itself.} The Athenians have
big incentives to do so, as Pericles himself mentioned this when praising the
city\sup{II.37}: \inlineQuote{public preferment depends on individual
distinction and is determined largely by merit rather than rotation}. This can
also be seen from Alcibiades: he is quite popular not because of his elite
background but his successful achievements which he boasted the first part of
his speech addressing the Sicily Expedition\sup{VI.16}. If such boast would do
something negative to him, a guy as intelligent as him would not do them in the
first place. His enemy also attacked him later\sup{VI.28} by accusing him
unpatriotic, another sign of how powerful honor was in ancient Greece.

An example from the other side—it’s dangerous not to “own” honor—is the comment
\sup{VI.24} of Thucydides after Nicias’ second speech on the magnitude of the
Sicily Expedition. Nicias had to keep quite before the public or he would be
branded as unpatriotic. In other words, one not only needs to do good for his
city, he also needs to convince others that he is really doing good for the
city. Moreover, under certain circumstances like the one Nicias was in, the
later is above the former in order to \emph{survive}. Due to the pressure of
this social norm, one cannot decide his own lifestyle—fight for his city or
simply enjoy his wealth—otherwise he will suffer significant pressure from
others. Such lack of freedom is extremely dangerous, especially in a war
setting, in which people may compete their honor and take extremely bold
actions. \emph{Therefore, honor has become a new type of private interest like
wealth, a tool to protect oneself and gain other personal interests.}
\end{\section{The norm of pursing “honor”}}

\bgn{\section{War’s effect}}
As mentioned above, such kind of honor is quite disastrous in a war setting. One
reason is that during a war, the relationship between the traditional types of
private interests like wealth and the externalized honor became even closer.
One could participate in expeditions and quickly gain lots of honor, which in
turn encourages doing so. This can be seen throughout the Athenians during the
war. At the very beginning\sup{I.144} of the war, Pericles had warned the
Athenians of not being overconfident. He had foreseen that the Athenians might
\inlineQuote{extend the  empire while the war is on” and might “undertake
additional risks of your own making.} The Athenian society is a place where one
could get infinite benefits through participating in expedition but could be
accused of unpatriotic even being merely conservative. Since the Athenians were
a quite capable military power, it’s not hard to imagine that expeditions
usually would end up with success and honor would be gained for participants.
Hence a “vicious circle” was created: from an individual level, one could see
his fellow citizens gain not only wealth but also honor from expeditions and
therefore become keen on participating those glorious activities in future; from
the city level, every single success strengthens the Athenians’ confidence, and
in the next assembly every one will have to pretend to be more confident, either
sincerely or for mere self-protection. It’s just like the nuclear competition
during the cold war. This was perfectly shown during the second assembly
\sup{VI.19 – 24} before the Sicilian expedition, in which Nicias exaggerated the
power of their potential enemies in order to deter the Athenians but the general
population becoming “yet more determined”.

The blending effects between war and honor is also somehow exhibited through the
different national characters between the Spartans and the Athenians. The
Spartans were said to be more \altvoice{different} and \altvoice{slow}, as
commented by the Corinthian representative\sup{I.70}, or \altvoice{prudent},
by Archidamus\sup{I.84}. In contrast, the Athenian national characters are
summarized as \altvoice{bold and adventurous}\sup{I.70}. I believe such
difference was resulted from the constitution of two cities. Sparta is an
oligarchic city ruled by only a few, which meant ordinary citizens lack much
incentive to be active in participating in city’s expedition. The Athenian
society worked more like a free market of honor: everyone is competing with each
other.

From this we come to the other potential negative effect of honor: that people
can have justification for their bold behaviors—they could console themselves
that what they are doing is for the sake of their city. If my argument was
right, we shall see more internal disturbance in democratic cities than
oligarchic ones, because the public have more freedom in doing what they think
is good for the whole city, and the honor norm gives them spiritual support.
Thucydides’ depiction and comments\sup{III.81 – 82} on the Corcyraean revolution
supports my point. Thucydides mentioned that many Corcyraeans were murdered when
Eurymedon stayed nearby with sixty ships\sup{III.81}. Most of them were killed
in the name of plotting to subverting the democracy—citizens were encouraged by
their faith that they were doing the just thing. One may argue that still some
Corcyraeans were killed because they have loaned money and in the later chaos
there were blatant robbery of wealth, and obviously people doing this for
personal interests only. But nevertheless the original faith that they are doing
good for their city did provide the Corcyraeans with the prerequisite of a
chaos, and once a chaos did take place, no one could control it. From this point
of view, honor as private interest is indeed dangerous.

The contrast example, ironically, came from the “Four Hundred revolution” in
Athens. After taking control of the city\sup{VIII.70}, the Four Hundred did not
create much chaos in Athens, and only a few were put to death. The oligarchy did
not have the interest in pursuing honor—private interests are all they need and
respect could be gained through direct force—while the ordinary citizens did not
have chance in doing good for the city spontaneously. Therefore, no collapse of
law happened while the Four Hundred was in charge.
\end{\section{War’s effect}}

\bgn{\section{Conclusion}}
My conclusion is as follows. The norm of pursing honor, though ensuring one’s
surface loyalty to his city, actually put the city in a more dangerous place.
The honor no longer stands as a moral standard, instead, it became a kind of
goods people, especially the lower class who did need them to gain other
personal interests but also the upper class who needed self protection from the
fury of the public, tried to obtain. Once the war started, such norm was quickly
intensified and led to bold actions. It affected the democratic cities most, as
the general public needs it more than the upper class. During civil disturbance,
honor could give citizens “excuses” to pursue their human nature, and usually
started chaos from which law started to collapse.
\end{\section{Conclusion}}
