\docclass{article}
\import{LiuMS.Texy.General}

\title{Second Essay}

This essay is about Plato’s works discussed in class, in particular,
\workname{Republic}. The prompt I choose from the three given by the professor
is:

\bgn{\quote}
How can the form of a city and the form of a soul be linked? How does Plato link
changes in the form or organization of a city with changes in the form or
organization of the soul? Is Plato also implying that any change in the just
city or soul always leads to injustice?
\end{\quote}

\@local{@aiWarnFn}
  \if {\priorToDate{2015-11-25}}
      {\id}
      {\deleted}
\@reflocal{@aiWarnFn}
  Prior to November 25th, any comments or advice on this essay is not encouraged
  for academic integrity reasons.

\hrule

In Book II of \workname{Republic}, Socrates proposed that in order to find the
justice of a man one should first find the justice of a whole city because
the latter is larger and therefore easier to find. After finding a city’s
justice, the rest effort lies at observing the similarity between a man and a
city.\sup{368e}

Then, an ideal city is constructed by adding the necessary part step by step.
First, a city requires ordinary men to farm, to craft, to maintain the material
life inside the city\sup{369a – 371c}. Second, a city requires guardians to
maintain its security from outside threats. The reason for separating guardians
from craftsmen is that in an ideal city one citizen should only specialize in
one field for maximum efficiency\sup{370c} and that warfare itself is a
profession\sup{374b}.

Next, Socrates said that such the city needs an overseer or ruler to preserve
its constitution\sup{412a}. He discussed in depth the responsibility and
eligibility of rulers. A ruler should be a judge who rules other souls with his
own just soul\sup{409a}. An ideal ruler should be isolated from injustice during
his childhood, or he could be easily deceived and become unjust\sup{409b}. Late
in his life he should learn injustice to be able to deal with them properly, and
since he had already built just models such unjust behaviors wouldn’t affect his
own soul\sup{409c}. Rulers should be chosen from guardians, as, in spite their
capability, they must devote themselves to the services of the whole city—as
Socrates argued in Book I—and guardians are the one best at loving their city
\sup{412c – d}. Hence, the three classes of an ideal city are created: rulers
who are the best of guardians, auxiliaries who help the rulers to maintain the
security and order of the city, and the rest craftsmen.

After establishing the three classes of a city, Socrates and his fellows
discussed justice inside the city, which is each class concentrates on its own
work\sup{433b}. Subsequently, he asserted that there exists three parts in a
soul, each linked to a class in the city\sup{439a – 441c}. The lowest part, the
appetitive part of a soul which drives a person to respond to thirst, hunger,
the desire for money and sex—“the so-called pleasures of the body”—is the
craftsmen class of a city\sup{442b}. The intermediate one is the spirited part
and is similar to the auxiliary guardians\sup{441e}. The highest part,
corresponding to the ruler class, is the rational part of a soul, capable of
rational calculation\sup{441e}. The spirited part is allied with the rational
part. For example, when one's appetite is in contrary to his rational
calculation—like reading \workname{Republic} versus going to parties—after
repetitively forcing himself to do the “right” rational thing, one will often
get very angry at himself\sup{440b}.

The link between the form of an ideal state and the form of a just man as
examined so far still seems to be a little vague: what is the specific role of
each class of the city—or each part of a soul—plays in different situations?
One way to look at this is through removing certain features of a class or even
the class itself entirely, and see the corresponding \emph{changes}. In Book
VIII, Plato discusses changes in the organization of a city and their
reflections in human beings. There are, according to his theories, four types of
inferior cities\sup{544c}. From the least worse to the most worse, they are
\altvoice{timocracy}, \altvoice{oligarchy}, \altvoice{democracy}, and
\altvoice{tyranny} respectively.

A city of \altvoice{timocracy}, according to Socrates, is a city where rulers
possess less wisdom\sup{547e} and the relationship between guardians and the
craftsmen is more like master and slave rather than equal friendship\sup{547c}.
Such decay is probably caused by a civil war started by some degraded rulers
\sup{547b}. Rulers are no longer pure, which makes wise people fearful. Those
not so wise guardians who should have been auxiliaries are chosen to lead,
making the whole city more war-oriented and honor-seeking\sup{547d – e}. At the
same time, the civil war introduces the idea of private properties to the
guardians, so even though they are not permitted to make money in public, they
will seek wealth in secret\sup{548b}. I believe the key change here is that
\emph{guardians now have desires accompanied by abandoning the total love of
philosophy}. Regarding a human’s soul, one’s rational part is partially unified
with his spirited part, though still maintaining absolute dominance over his
appetite. The man will seek honor and victory, being rude to his subordinates
while being obedient to his superiors\sup{548e – 549a}. This should exemplify
the importance of rulers’ unquestionable love of philosophy as well as a
person’s “zeal” for wisdom. I think the modern correspondence to a timocratic
person would be those successful in different fields, like successful
businessmen: they honor victory, maintains a good control to their desires—no
procrastination for example—and are more “practical”.

Continuing decay leads to \altvoice{oligarchy}. This is caused by the unchecked
growth of the guardians’ power as well as their thirst for wealth\sup{550d}.
There are actually no more difference between rulers and auxiliaries, only the
rich and the poor, the ruling and the ruled. In terms of a man, it is a soul
which is fully devoted to making money, or, putting into the modern world,
material interests. The first fault of oligarchy is that everyone except the
poor can become a ruler\sup{551c}. Everything except instant interest can become
the goal of an oligarchic soul. Such a soul is \emph{short-sighted}
\sup{554a – b}. The second fault of oligarchy is the existence of a head-on
conflict between the ruler and the ruled\sup{551d}. Translating this fault to
desires in a human being, this says that such a person no longer has a full
control over his desires and has to rely on self-correction over and over again
\sup{554c}. I believe this shows the importance of a standalone ruling class,
the separation between the rational part and the spirited part. Unlike a man of
excellence or a timocratic person who fights for the wisdom or victory itself
rather than their accompanying results, an oligarchic person seeks solely for
the material interests. This change leads to the failure of building an
“institution” of systematic control of one’s appetite, or as Plato said in the
book, “such a man pays no attention to eduction”\sup{554b}. I believe oligarchy
reflects the “soul”, or mental status, of average successful person.

While an oligarchy is wealth-oriented, a \altvoice{democracy} is much more
diversified. There are no longer two different classes, and everyone—mostly the
original craftsmen class—now have the freedom to do whatever they want to do
\sup{557b}. The very idea of founding an ideal state—specialization—is now
replaced by chaos and arbitrariness\sup{558b}. The corresponding changes in
one’s soul makes this person surrounded by desires that are both necessary and
redundant\sup{558e}. A man of this kind doesn’t have a very focused objective
and pursue whichever desire comes into his mind\sup{561b}, which is often called
lack of self-discipline. In the other hand, such lifestyle is often envied by
other people\sup{559e}, providing that the person in question could figure out
a way to ensure his material life quality. The democracy constitution shows how
the absence of guardians, or both the rational part as well as the spirited part
, could impact a city or a person. Arbitrary desires now take over. I would say
those unsuccessful man who shows a clear lack of self-control falls into this
category.

The last one, \altvoice{tyranny}, is quite similar to democracy from a certain
point of view. The evolution is simple: the rich people are robbed by the rest
majority\sup{564e – 565c} and a leader will be elected by the majority for the
effective opposition of the rich, who are starting to strike back\sup{565d}. The
tyranny’s power grows stronger and stronger and starts to purge all enemies
including those who have previously supported him, especially those who are
brave and well educated—the good citizens\sup{567c – d}. At one point, no one
could no longer challenge his position, and relatively capable citizens have
been purged. In the ninth book, Socrates provided a model for a tyrannic man to
develop. Initially, both the moderate desires and the lawless ones are developed
during the child’s imitation of others, particularly his parents\sup{572e}.
However, with the help of bad companies the child gradually replaces these good
desires with those crazy, “erotic” ones. This is a process like the rise of a
tyrant, which, once started, can hardly be reverted\sup{573a – b}. To
instantiate this model with some real world examples, one can think these bad
desires as gaining wealth through theft or gaining pleasure through drugs. Once
addicted, they can hardly be shaken off. As Socrates showed\sup{575b}, one good
example of tyrannic people is criminals.

Clearly, Plato implied that any change to the just city leads to injustice, as
shown by the usage of words like “decay”\sup{546a}. In Book IV, Plato already
defined “justice” as everyone doing his own work, as mentioned above\sup{433b}.
Translated into the context of human souls, this says justice is and only is the
mental state that one is guided by his rational part allied with the spirited
part and has a tame appetitive part. Each change in the form of city mentioned
above pulls one class from what it is best at and therefore disturb the order of
a city, which leads to injustice in a city. Similarly, each change in the
organization of soul disorders the soul itself, which also results in injustice.
