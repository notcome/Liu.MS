\warning{2016-09-01}{更新就读年级}
我现在是一名大一学生,就读于加州大学圣迭戈分校。我的专业是语言学,计划第二专业数学、辅修政治科学。

我的母语是\link{@mandarin}。英语是我的第二语言,但掌握不够熟练:词汇——尤其是生活词汇和生僻词
汇——掌握较少,口语能力较差。我并不会其它语言。

\local{@mandarin}{\mkLink{官话}{https://zh.wikipedia.org/wiki/官话}}

学术方面,我主要对计算语言学和程序语言理论比较感兴趣。具体到动机,我是\link{@strongAI}的支持
者,希望提高机器对语言处理与理解的能力,并相信先进的编程语言是复杂的软件项目如人工智能的基石。我
没有选择在本科就读计算机科学专业,有如下原因:
\@ulist
  \@item 我已有多年编程经验\refSideNote{@csExp},有算法竞赛的基础,可以自学计算机科学相关
    的知识。
  \@item 本校的学分限制。我本科四年最多只能学习六十门课,如果我选择计算机科学专业,我将失去学
    习数学、政治科学、或其它人文科学的机会。
  \@item 我较为懒散。为了\link{@nlp}去学习机器学习,或者为了编程去学习\link{@archery},这
    些「刚需」我还勉强能满足。可是,又有什么刚需让我去读古代经典或者高阶的数学呢?

\local{@strongAI}{\mkLink{强人工智能}{https://zh.wikipedia.org/wiki/强人工智能}}
\local{@nlp}{\mkLink{自然语言处理}{https://zh.wikipedia.org/wiki/自然语言处理}}
\local{@archery}{\mkLink{范畴论}{https://zh.wikipedia.org/wiki/范畴论}}
\local{@csExp}{\mkSideNote{我没有多少大项目的经验,更多的都是小玩具与原型。}}

如上所述,我较为懒散。本网站建立的动机之一便是希望实时更新我的工作进度,通过发布至互联网产生有人
监视的错觉,提高我的工作效率。

我对界面设计比较感兴趣。此处的「界面」不仅仅指图形用户界面,它包括了从微波炉按钮到网页组织结构的
一切广义的界面。不过,我主要的兴趣还是在图形用户界面上。

目前我最喜欢的英文衬线体是\link{@charter}。不过我并不觉得它算真正意义的衬线体,其衬线粗且平直,
并不像其它经典的衬线体一样有尖锐的衬线。我更多地是将其看作一种有着衬线作视觉补偿的无衬线体。

\local{@charter}{\mkLink{Charter}{https://en.wikipedia.org/wiki/Bitstream_Charter}}

我喜欢讲一些奇诡的笑话,被朋友评为「恶意卖萌」、「爱讲冷笑话」。
